\documentclass[12pt, a4paper, english, russian]{article}

\usepackage[margin=1.9cm]{geometry}
\usepackage{fontspec, babel,
            indentfirst, ulem,
            mathtools, tocloft,
            xurl, amsfonts,
            hyperref, float,
            listings, xcolor,
            graphicx, caption,
            subcaption, enumitem,
            ragged2e, hyphenat,
            csquotes, pgf}
\usepackage[
    parentracker=true,
    backend=biber,
    hyperref=auto,
    language=autobib,
    autolang=other,
    style=gost-numeric,
    sorting=none
]{biblatex}

\sloppy

\setmainfont{Times New Roman}
\setcounter{secnumdepth}{0}

\setlength{\cftbeforetoctitleskip}{0em}
\renewcommand{\cftsecleader}{\cftdotfill{\cftdotsep}}
\renewcommand\cftsecfont{\mdseries}
\renewcommand\cftsecpagefont{\mdseries}

\lstset{
    showspaces=false,
    showstringspaces=false,
    showtabs=false,
    breakatwhitespace=false,
    breaklines=true,
    keepspaces=true,
    texcl=true
}

\hypersetup{
    colorlinks=true,
    citecolor=black,
    filecolor=black,
    linkcolor=black,
    urlcolor=blue,
    breaklinks=true
}

\raggedbottom
\setlist[enumerate]{noitemsep,topsep=0pt}
\setlist[itemize]{noitemsep,topsep=0pt}

\addbibresource{refs.bib}

\newcommand{\pic}[2]{
	\begin{figure}[H]
		\centering
		\includegraphics[width=0.95\linewidth]{#1}
		\caption{#2}
	\end{figure}
}

\begin{document}

\begin{titlepage}
	\begin{center}
		{\fontsize{11}{13}
			Федеральное государственное бюджетное образовательное учреждение высшего образования

			«Сибирский государственный университет телекоммуникаций и информатики»

			(СибГУТИ)
		}
	\end{center}
	\begin{center}
		\begin{large}
			Институт информатики и вычислительной техники
		\end{large}
	\end{center}

	\begin{flushright}
		\begin{minipage}[t]{0.59\textwidth}

			\uline{09.04.01 ''Информатика и вычислительная техника''}

			\uline{профиль ''Научные исследования в области}

			\uline{информатики и вычислительной техники''}

		\end{minipage}
	\end{flushright}

	\vspace{0.012\textheight}

	\begin{center}
		\begin{large}
			Кафедра прикладной математики и кибернетики
		\end{large}
	\end{center}

	\vspace{0.1\textheight}

	\begin{center}
		\begin{Large}
			\textbf{Лабораторная работа №5}

			\textbf{по дисциплине}

			\textbf{Прикладная стеганография}

		\end{Large}
	\end{center}

	\vspace{0.12\textheight}

	\begin{flushleft}
		\begin{large}
			Выполнил:
			\vspace{0.012\textheight}

			студент гр.МГ-411 \hfill $\underset{\text{ФИО студента}}{\text{Каргин Роман Александрович}}$

			«17» апреля 2025 г.

		\end{large}
	\end{flushleft}

	\vfill

	\begin{center}
		\begin{large}
			Новосибирск 2025 г.
		\end{large}
	\end{center}

\end{titlepage}

\tableofcontents

\pagebreak

\section{Задание}

\begin{enumerate}
	\item Составить обзор статистических методов стегоанализа изображений:  анализ статистики Хи-квадрат, RS-анализ, метод AUMP.
	\item Реализовать программное средство для стегоанализа изображений, включающее в себя:
	      \begin{enumerate}
		      \item Визуальную атаку на стегоконтейнер, взятую из задания №1;

		      \item Анализ статистики Хи-квадрат по частям изображения;

		      \item RS-анализ, взятый из источника:  \url{https://github.com/b3dk7/StegExpose/blob/master/RSAnalysis.java}

		      \item Метод AUMP, взятый из источника: \url{http://dde.binghamton.edu/download/structural_lsb_detectors/}

		      \item Дополнительно можно реализовать стегоанализ на основе сжатия.
	      \end{enumerate}
\end{enumerate}
Необходимо, чтобы программа позволяла загружать как отдельное изображение,
так и сразу несколько изображений, предоставив пользователю возможность выбрать расположение файлов.

Результаты стегоанализа должны отображаться в интерфейсе программного средства
в понятном для пользователя виде, предполагая работу стороннего стегоаналитика.
При анализе нескольких файлов сразу, результаты должны записываться в текстовый
файл по выбранному пути сохранения.

Отчет по работе должен содержать результаты всех пунктов задания, включая описание кода программы.

\pagebreak

\subsection{Имплементация}
Ссылка на код --- \url{https://github.com/Nulliream/steg/tree/main/Task5}

Все алгоритмы реализованы в модуле control. В качестве библиотеки для
обработки изображений использовался PILLOW.

Интерфейс --- в модуле ui. В качестве библиотеки дял интерфейса использовался Qt.

\subsubsection{$\mathbf{\chi^2}$}
$\chi^2$ анализ используется для определения вероятности того, что оба распределения,
основываясь на большой выборке данных, относятся к одному типу распределения.

В качестве сравниваемых выборок мы берём значения младших трёх битов: 000, 001 и т.д.
Мы попарно нормализируем блоки ряда: 000 и 001, 010 и 011, 100 и 101, 110 и 111,
суммируя значения и деля их на два. Данные два распределения мы и сравниваем.

Код алгоритма предоставлен в классе Chi2. В логике реализовано вручную
лишь построение выборок и их нормализация. Для выполнения теста $\chi^2$
используется функция пакета scipy со степенью свободы 1.

\subsubsection{RS}
Изображение делится на группы соседних пикселей размером n. Определяется
функция, принимающая в качестве аргумента пиксели группы и возвращающая некоторое
вещественное число, описывающее их схожесть друг с другом ($f(x)$). Определяется функция,
переворачивающая младший бит пикселя ($F(x)$).

Определяются три группы:
\begin{gather*}
	\text{Регулярная: } G \in R \Leftrightarrow f(F(G)) > f(G) \\
	\text{Сингулярная: } G \in S \Leftrightarrow f(F(G)) < f(G) \\
	\text{Неиспользуемая: } G \in U \Leftrightarrow f(F(G)) = f(G)
\end{gather*}
где $F(G) = (F(x_1),\ldots,F(x_n))$. Чтобы решать, какие пиксели переворачивать,
используется маска $M$, являющаяся массивом из значений $-1$, $0$, $1$.

Пусть $R_M$ --- число регулярных групп для маски $M$, $S_M$ --- число сингулярных
групп для маски $M$. По условиям $R_M + S_M \leq 1$ и $R_{-M} + S_{-M} \leq 1$
мы можем предположить, что в обычном изображении $R_M \cong R_{-M}$ и $S_M \cong S_{-M}$.
При увеличении вложенного сообщения разница между $R_M$ и $S_M$ уменьшается,
а между $R_M$ и $R_{-M}$, $S_M$ и $S_{-M}$ увеличивается. Основываясь на этом,
можно вычислить вероятный размер вложенного сообщения.

\subsubsection{AUMP}
Пиксели собираются в группы размером $n$ каждый, чтобы их распределение
зависело от малого числа параметров.
После тестируется гипотеза о том, что распределения без спрятанных битов
и со спрятанными совпадают.

\pagebreak

\subsection{Результаты}
\pic{images/app.png}{Главное окно программы}
\pic{images/opening_images.png}{Открытие изображений}
\pic{images/attack_images.png}{Визуальная атака}
\pic{images/analysis.png}{Анализирование}
\pic{images/saving.png}{Сохранение результатов}
\pic{images/saves.png}{Сохранённые результаты}

\end{document}
