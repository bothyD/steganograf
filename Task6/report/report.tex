\documentclass[12pt, a4paper, english, russian]{article}

\usepackage[margin=1.9cm]{geometry}
\usepackage{fontspec, babel,
            indentfirst, ulem,
            mathtools, tocloft,
            xurl, amsfonts,
            hyperref, float,
            listings, xcolor,
            graphicx, caption,
            subcaption, enumitem,
            ragged2e, hyphenat,
            csquotes, pgf}
\usepackage[
    parentracker=true,
    backend=biber,
    hyperref=auto,
    language=autobib,
    autolang=other,
    style=gost-numeric,
    sorting=none
]{biblatex}

\sloppy

\setmainfont{Times New Roman}
\setcounter{secnumdepth}{0}

\setlength{\cftbeforetoctitleskip}{0em}
\renewcommand{\cftsecleader}{\cftdotfill{\cftdotsep}}
\renewcommand\cftsecfont{\mdseries}
\renewcommand\cftsecpagefont{\mdseries}

\lstset{
    showspaces=false,
    showstringspaces=false,
    showtabs=false,
    breakatwhitespace=false,
    breaklines=true,
    keepspaces=true,
    texcl=true
}

\hypersetup{
    colorlinks=true,
    citecolor=black,
    filecolor=black,
    linkcolor=black,
    urlcolor=blue,
    breaklinks=true
}

\raggedbottom
\setlist[enumerate]{noitemsep,topsep=0pt}
\setlist[itemize]{noitemsep,topsep=0pt}

\addbibresource{refs.bib}

\newcommand{\pic}[2]{
	\begin{figure}[H]
		\centering
		\includegraphics[width=0.95\linewidth]{#1}
		\caption{#2}
	\end{figure}
}

\begin{document}

\begin{titlepage}
	\begin{center}
		{\fontsize{11}{13}
			Федеральное государственное бюджетное образовательное учреждение высшего образования

			«Сибирский государственный университет телекоммуникаций и информатики»

			(СибГУТИ)
		}
	\end{center}
	\begin{center}
		\begin{large}
			Институт информатики и вычислительной техники
		\end{large}
	\end{center}

	\begin{flushright}
		\begin{minipage}[t]{0.59\textwidth}

			\uline{09.04.01 ''Информатика и вычислительная техника''}

			\uline{профиль ''Научные исследования в области}

			\uline{информатики и вычислительной техники''}

		\end{minipage}
	\end{flushright}

	\vspace{0.012\textheight}

	\begin{center}
		\begin{large}
			Кафедра прикладной математики и кибернетики
		\end{large}
	\end{center}

	\vspace{0.1\textheight}

	\begin{center}
		\begin{Large}
			\textbf{Лабораторная работа №6}

			\textbf{по дисциплине}

			\textbf{Прикладная стеганография}

		\end{Large}
	\end{center}

	\vspace{0.12\textheight}

	\begin{flushleft}
		\begin{large}
			Выполнил:
			\vspace{0.012\textheight}

			студент гр.МГ-411 \hfill $\underset{\text{ФИО студента}}{\text{Каргин Роман Александрович}}$

			«21» апреля 2025 г.

		\end{large}
	\end{flushleft}

	\vfill

	\begin{center}
		\begin{large}
			Новосибирск 2025 г.
		\end{large}
	\end{center}

\end{titlepage}

\tableofcontents

\pagebreak

\section{Задание}

Написать программу, которая использует выходные текстовые файлы из программного средства стегоанализа, реализованного в задании №5 для подсчета ошибки 1 и 2 рода. Заполненные контейнеры взять из результатов работ №3-4.

Отчет по работе должен содержать описание формата вывода данных стегоанализа, подсчет ошибки 1 и 2 рода и таблицу сравнения методов стегоанализа. В таблице привести результаты стегоанализа при разном заполнении контейнеров, указав максимально возможную фактическую ёмкость контейнера и \% заполнения стегоконтейнера. Например, 50\% заполненный стегоконтейнер при последовательном заполнении; при рассеянном заполнении; также для 100\%. Привести ссылку на исходники.
\pagebreak

\subsection{Результаты}
Код --- \url{https://github.com/Nulliream/steg/tree/main/Task6}

Ниже показаны результаты замеров. По X обозначен процент заполнения контейнера. По Y --
процент ошибок первого/второго рода в зависимости от наличия сообщения в проверяемом контейнере..
\pic{images/seq-20.png}{Сравнение последовательного заполнения для Chi2, AUMP, RS для 20 изображений}
\pic{images/lab3-20.png}{Сравнение заполнения из лабораторной 3 для Chi2, AUMP, RS для 20 изображений}
\pic{images/seq-100.png}{Сравнение последовательного заполнения для Chi2, AUMP для 100 изображений}
\pic{images/lab3-100.png}{Сравнение заполнения из лабораторной 3 для Chi2, AUMP для 100 изображений}
\pic{images/lab4-100.png}{Сравнение заполнения из лабораторной 4 для Chi2, AUMP для 100 изображений}

\begin{table}[H]
	\centering
	\begin{tabular}{c|c|c|c}
		Заполнитель & Тип ошибки & Детектор & Процент \\
		\hline
		Послед      & I          & Chi2     & 60      \\
		Послед      & I          & AUMP     & 30      \\
		Послед      & I          & RS       & 35      \\
		\hline
		Послед      & II         & Chi2     & 35      \\
		Послед      & II         & AUMP     & 15      \\
		Послед      & II         & RS       & 0       \\
		\hline
		Lab3        & I          & Chi2     & 60      \\
		Lab3        & I          & AUMP     & 30      \\
		Lab3        & I          & RS       & 35      \\
		\hline
		Lab3        & II         & Chi2     & 25      \\
		Lab3        & II         & AUMP     & 10      \\
		Lab3        & II         & RS       & 20      \\
		\hline
		Lab4        & I          & Chi2     & 80      \\
		Lab4        & I          & AUMP     & 20      \\
		Lab4        & I          & RS       & -       \\
		\hline
		Lab4        & II         & Chi2     & 20      \\
		Lab4        & II         & AUMP     & 80      \\
		Lab4        & II         & RS       & -
	\end{tabular}
	\caption{Таблица сравнений}
\end{table}

\end{document}
